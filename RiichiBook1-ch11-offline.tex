%~~~~~~~~~~~~~~~~~~~~~~~~~~~~~~~~~~~~~~~~~~~~~~~~~
% Riichi Book 1, Chapter 11: offline
%~~~~~~~~~~~~~~~~~~~~~~~~~~~~~~~~~~~~~~~~~~~~~~~~~

\chapter{Manners for offline playing} \label{ch:manners}
\thispagestyle{empty}

Manners are meant to make the game of mahjong a pleasant experience. 
They are a collection of small tips and techniques the forerunners have developed to avoid unnecessary troubles. 
I present manners for four different phases of a game --- (1) dealing tiles, (2) drawing and discarding, (3) calling, and (4) winning a hand. 
Each entry is given a rank, from one star ($^*$) to three stars ($^{***}$). Three-star manners are more important; you should try to acquire three-star manners first, and then move on to practice two-star and one-star ones. 

\section{Dealing tiles}

\subsection*{1-1: Shuffling $^{**}$}
I recommend a 2-step shuffling approach. First, shuffle tiles really hard. Don't worry about keeping the tiles face down at this point. This will guarantee that sets, runs, and pairs from the previous hand are really broken apart. Second, put all the tiles up-side down and shuffle them face down gently. This will guarantee that no one remembers the locations of certain tiles. 

\subsection*{1-2: Push the wall forward $^{**}$}
Once you build a wall, push it forward a little so that the facing player can easily reach your wall. If you push it forward too much, you will lose the space for discards. 


\subsection*{1-3: Tilt the wall $^{*}$}
When pushing the wall forward, it would be better if you tilt the wall a little, as follows. This will make it even easier for the facing player to pick a tile from your wall.
\bp
\rotatebox{4}{
\raisebox{-0.5ex}[0pt][0pt]{
\Large \suo{30}\suo{30}\suo{30}\suo{30}\suo{30}\suo{30}%
\suo{30}\suo{30}\suo{30}\suo{30}\suo{30}%
\suo{30}\suo{30}\suo{30}\suo{30}\suo{30}\suo{30}
	}}
\ep

\subsection*{1-4: Split the wall $^{*}$}
In addition to tilting, some players like to mildly split the wall into three blocks upon building the wall. You get six tiles on the left, five tiles in the middle, and another six tiles on the right, illustrated as follows. 
\bp
\rotatebox{4}{
\raisebox{-0.5ex}[0pt][0pt]{
\Large \suo{30}\suo{30}\suo{30}\suo{30}\suo{30}\suo{30}}%
\Large \suo{30}\suo{30}\suo{30}\suo{30}\suo{30}%
\raisebox{0.5ex}[0pt][0pt]{\Large\suo{30}\suo{30}\suo{30}\suo{30}\suo{30}\suo{30}}
}
\ep
This will make it significantly easier for the dealer to identify the breaking point in the wall. 

\subsection*{1-5: Break the wall $^{*}$}
After rolling the dice, the dealer should break the wall himself. When the dice indicate a number $k$ that is greater than 7, it is easier to count $15-k$ tiles counterclockwise rather than counting $k$ tiles clockwise \emph{after} identifying which wall to break. 
For example, when the dice roll is 9, the dealer should count 6 tiles counterclockwise (i.e., 6 from the left edge), leaving 6 tile pairs on the left of his wall. Likewise, when the dice roll is 10, the dealer should count 5 tiles counterclockwise, leaving 5 tile pairs on the right of the right player's wall (right from the dealer's view, left from the right player's view). 

\subsection*{1-6: Put the {\jap rinshan} tile down $^{**}$}
After breaking the wall, the {\jap rinshan} tile (the first replacement tile) should be preemptively put down. This is to prevent it from falling over. This should be done by the player who has the dead wall in front of him. 

\subsection*{1-7: Turn over the {\jap dora} indicator $^{***}$}
Immediately after putting down the {\jap rinshan} tile, the {\jap dora} indicator should be turned over. Doing so is \emph{way} more important than, say, separating the dead wall from the end of the wall (which is completely unnecessary especially at the beginning of a hand). In my experience, European players somehow like to do the latter first and don't open the {\jap dora} indicator even after they finish the dealing. 

\subsection*{1-8: Look at the tiles $^{**}$}
As you take tiles from the wall during the initial dealing, you should start taking a look at them. Don't wait until you get all thirteen tiles; doing so is a waste of time not only for you but also for the other players. 

\subsection*{1-9: Dealer's first discard $^{**}$}
The dealer should not discard a tile until the North player gets all thirteen tiles. This is to give everyone a roughly equal amount of time to decide whether to call {\jap pon} / {\jap chii} / {\jap ron} on the first discard.

\section{Drawing and discarding}

\subsection*{2-1: Don't use both hands $^{**}$}
During the play --- after the dealing is done and before the hand finishes ---, you should use only one hand. If you are right-handed, you should not put your left hand on the table, either. Don't do things like drawing with your left hand and discarding with your right hand. This is to prevent (the appearance of) cheating. 
The only occasions where using both hands during the play is acceptable are (1) when sorting the tiles in your hand, and (2) when revealing your hand upon winning or in cases of exhaustive draw and four-riichi abortive draw. 

\subsection*{2-2: Arrange the discards $^{***}$}
Discards should be arranged in an orderly way (six tiles in a row). 

\subsection*{2-3: Discard before sorting $^{*}$}
You should discard a tile before you put the tile you draw into your hand. Putting the newly drawn tile into your hand upon {\jap tsumo} is a serious violation. To avoid it, you should make a habit of not putting a newly drawn tile into your hand immediately. 

\subsection*{2-4: Let go of the discard $^{**}$}
Upon discarding a tile, you need to let it go immediately and not keep your finger on the tile. This is to guarantee that all the other three players can see which tile you discarded all together.

\subsection*{2-5: Don't take an overly long time $^{*}$}
Keep in mind that three other players are waiting for you; you should not take an overly long time to draw / discard. In particular, beginners may want to pay attention to the following. 
\bi\itemsep.1pt
\i When your turn comes, draw a tile immediately (unless you need to think about whether to call the last discarded tile). 
\i Once you make up your mind about what to discard, discard it immediately.
\ei

\subsection*{2-6: Don't speed other players $^{***}$}
Yes, it could be irritating if someone is taking a long time, but be considerate.
You should not press other players to be faster than they could.

\section{Calling}

\subsection*{3-1: Vocalize clearly $^{***}$}
When you call {\jap pon} \textipa{[p\'\textopeno\ng]}, {\jap chii} \textipa{[t\textesh\'\textsci\textlengthmark]}, {\jap kan} \textipa{[k\'\textturnv\ng]}, riichi \textipa{[r\'\textsci\textlengthmark t\textesh]}, {\jap ron} \textipa{[r\'\textopeno\ng]}, or {\jap tsumo} \textipa{[ts\'umo]}, utter the word clearly so the other three players can hear you. 

\subsection*{3-2: Vocalize before taking an action $^{***}$}
When calling {\jap pon}, say ``{\jap Pon}.'' first before taking the tile. Likewise, when you call riichi, say ``{\jap Riichi}.'' first before discarding a tile and placing a riichi bet (see also 3-4 below). 

\subsection*{3-3: Wait before calling {\jap chii} $^{*}$}
When calling {\jap chii}, wait for 1 second before uttering the word. On the other hand, when calling {\jap pon} / {\jap kan} / {\jap ron} you must do so immediately. If someone says {\jap chii} first (after taking 1 second), other players should not be able to call {\jap pon} / {\jap kan} / {\jap ron}. {\jap Pon}, {\jap kan}, and {\jap ron} should take precedence \emph{only if} calls are made concurrently.\footnote{EMA rules allow a {\jap pon} call to occur even after a {\jap chii} call is made. I think this should be changed.}

\subsection*{3-4: Calling riichi $^{***}$}
The procedure to call riichi is as follows.
\be\itemsep.0pt
\i Say ``{\jap Riichi}.''
\i Discard a tile, rotating it sideways.
\i Confirm that no one calls {\jap ron} on the discarded tile.
\i Place a riichi bet.
\ee
The most important point is that you say ``{\jap Riichi}.'' before discarding a tile. This is because the opponents' choice of what to do with your discard (i.e., whether or not to call {\jap pon} on it, etc.) may be different if you riichi.

\section{Winning a hand}

\subsection*{3-1: Vocalize clearly $^{***}$}
When winning a hand, you need to say {\jap ron} or {\jap tsumo} clearly. It is also OK to say ``mahjong'' instead. 

\subsection*{3-2: Don't put the winning tile into the hand $^{***}$}
When winning by {\jap tsumo}, don't place the winning tile inside the hand. Just place the winning tile right next to your hand. This is important because scores ({\jap yaku} and minipoints) may be different depending on which tile was the one to complete the hand. 

\subsection*{3-3: Don't take the winning tile $^{*}$}
When winning by {\jap ron}, some European players grab the winning tile and place it right next to their hand. Don't do it. You should refrain from doing this to prevent (the appearance of) cheating. People do this on TV, but they do so only for the camera. 

\subsection*{3-4: Sort the tiles before revealing your hand $^{**}$}
You need to sort the tiles before showing your hand, so that other players can easily check your hand's score and possible {\jap furiten} violation. Do not split the hand into constitutive groups. Doing so may actually obstruct other players' vision. 

\subsection*{3-5: Declare {\jap yaku} $^{**}$}
After revealing your hand, reveal the {\jap ura dora} if you have called riichi. You need to show the {\jap ura dora} to all the other players even when you don't get any of them. This is to make sure that you are not underreporting your hand value.\footnote{It may sound odd, but there are situations where you have strategic incentives to underreport your hand value. Trying to avoid bankruptcy of another player when you are still ranked second or third is one obvious example. For another example, players may not want to change the placements of other players in a game if they are competing for ranking at a tournament. Underreporting the hand value is usually illegal.}
After that, you should declare all the {\jap yaku} in your hand.\footnote{Some people may say that you only need to declare the score and that declaring {\jap yaku} is either unnecessary or even undesirable. I personally think it's unnecessary, especially when playing with experienced players. However, given that not everyone at the table can quickly identify all the {\jap yaku} in another player's hand, it would be prudent if the winner declares all the {\jap yaku}.}

\subsection*{3-6: Declare the score $^{***}$}
You need to declare the score of your hand yourself. 
It is OK to get other players' help on scoring, but you need to be the one to declare it. When declaring {\jap tsumo} scores, say the payment by a non-dealer first, followed by the payment by the dealer. For example, when declaring a 300-500 {\jap tsumo}, say ``Three hundred, five hundred.'' rather than ``Five hundred, three hundred.'' 

\subsection*{3-7: Confirm the score $^{*}$}
When one player wins a hand, the other three players must also see the hand and confirm the declared score. You should also check if the hand was not {\jap furiten}. 

\subsection*{3-8: Payment $^{*}$}
A standard stick set would include four kinds of sticks, as follows.

\bigskip
\begin{tabular}{l r l r}
\includegraphics[width=1in]{figs/tenbou100} & 100 point & 
\includegraphics[width=1in]{figs/tenbou1000} & 1000 point \\
\includegraphics[width=1in]{figs/tenbou5000} & 5000 point & 
\includegraphics[width=1in]{figs/tenbou10000} & 10000 point \\
\end{tabular}

%\bigskip
%\begin{tabular}{l r l r}
%\rotatebox{90}{\includegraphics[height=1in]{figs/100thin}} & 100 point & 
%\rotatebox{90}{\includegraphics[height=1in]{figs/1000thin}} & 1000 point \\
%\includegraphics[width=1in]{figs/tenbou5000} & 5000 point & 
%\includegraphics[width=1in]{figs/tenbou10000} & 10000 point \\
%\end{tabular}

\bigskip \noindent
In addition to these, I suggest you prepare a set of four 500-point sticks. I usually use green 100-point sticks that I bought in Japan, which look like: \rotatebox{90}{\includegraphics[height=1in]{figs/tenbou500}}.\footnote{The image of 500-point stick was created by someone known as ``381654729'' ({\jap Tenhou} ID: 零面聴). I thank him for letting me use it in this book.} If you don't have any green sticks, you can use anything (e.g., coins, poker chips, etc.) for substitute. 
Having 500-point sticks would make stick payment much more efficient.

\bigskip
To streamline the payment, you should try to minimize the number of sticks exchanged on the table. Here are two examples of efficient method of payment. 

\subsection*{3900 {\jap ron}}
When a player wins a 3900 hand, the player who discarded the winning tile gives the winner one \includegraphics[width=1in]{figs/tenbou5000}, and the winner gives  back 1100 points with one \includegraphics[width=1in]{figs/tenbou1000} and one \includegraphics[width=1in]{figs/tenbou100}. 

\subsection*{5200 {\jap tsumo}}
When a player gets a 1300-2600 {\jap tsumo}, the most efficient and beautiful method of payment is as follows.
\be
\i The first non-dealer (the one sitting closer to the winner) pays the exact amount with one \includegraphics[width=1in]{figs/tenbou1000} and three \includegraphics[width=1in]{figs/tenbou100}.
\i The second non-dealer gives the winner 1500 with\\
one \includegraphics[width=1in]{figs/tenbou1000} and one \rotatebox{90}{\includegraphics[height=1in]{figs/tenbou500}}.
\i The winner gives back the second non-dealer 200 with\\
two of the three \includegraphics[width=1in]{figs/tenbou100} he got from the first,
which ensures that the second non-dealer pays 1300.
\i The dealer gives the winner 5100 with one \includegraphics[width=1in]{figs/tenbou5000} and one \includegraphics[width=1in]{figs/tenbou100}.
\i The winner gives back the dealer 2500 with the two \includegraphics[width=1in]{figs/tenbou1000} he got from the two non-dealers and the one \rotatebox{90}{\includegraphics[height=1in]{figs/tenbou500}} he got from the second non-dealer, which ensures that the dealer pays 2600.
\i After all the exchanges, what remains on the table is exactly 5200 with one \includegraphics[width=1in]{figs/tenbou5000} and two \includegraphics[width=1in]{figs/tenbou100}.
\ee
For this to work out perfectly, everyone needs to be on the same page. It may sound complicated at first, but it sure feels good when the four players manage to make it happen together. 

\subsection*{3-9: Exhaustive draw $^{*}$}
In case of exhaustive draw, the dealer should be the first one to declare whether or not he has a ready hand. If he wants to declare ready, he has to show the hand and say ``{\jap Tenpai}.''; if not, say ``{\jap Noten}.'' or ``Not {\jap tenpai}.'' without showing his hand. Then, South, West, and North declare {\jap tenpai} or {\jap noten} in that order. 

\bigskip
The order of declaration could make a difference in some (rare) occasions. Declaring first is advantageous in some instances and disadvantageous in others. Suppose the dealer is ranked first in South-4, having 2900 more points than the second ranked player. Suppose further that he has a ready hand. In such a situation, the dealer has an incentive to make a declaration \emph{after} the second ranked player. If the second ranked player declares {\jap noten}, the dealer would want to declare {\jap noten} to terminate the game.\footnote{Continuing the game means he runs the risk of losing the placement bonus (10000 in EMA rules) and the {\jap oka} points (if any).} On the other hand, if the second ranked player declares {\jap tenpai}, the dealer would want to declare {\jap tenpai} and continue the game. This is because the induced point difference in case of {\jap tenpai}--{\jap noten} is either 3000 (2-player {\jap tenpai}) or 4000 (1-player {\jap tenpai} or 3-player {\jap tenpai}), each of which exceeds the current point difference of 2900. 

\bigskip
Declaring first can be advantageous only when playing with a bankruptcy rule. Suppose one player is at the verge of bankruptcy, having only 1300 points. Suppose further that both he and the first ranked player have already declared {\jap noten}. 
In such a situation, if the second ranked player declares {\jap tenpai} first, the third ranked player would have to declare {\jap noten} even when he has a ready hand. Otherwise, the fourth ranked player goes bankrupt and the game is terminated.

\bigskip
Because of these advantages and disadvantages of declaring first, we should stick with the predetermined order for the sake of fairness. 

\newpage

\begin{comment}
\begin{boxnote} \small
{\bf\normalsize Notes on EMA's three-second rule} \index{european@EMA}
\bigskip

As I mentioned, EMA rules allow a {\jap pon} call to occur within three seconds, even after someone calls {\jap chii} already. This rule opens the door to two kinds of dirty tactics, and hence it should be abolished. 

\bigskip
First, this allows players to intercept a {\jap chii} call just to balk another player. This may sound trivial, but it is really not. Suppose someone calls {\jap chii} during the last turn of a hand. Then, it is highly likely that he is turning his 1-Away hand into a ready hand. If you have a 1-Away or worse hand, you could intercept this {\jap chii} by calling {\jap pon}. The beauty of this tactics is that you don't need to have the correct Pair to call {\jap pon} on the discarded tile. You can still legally claim {\jap pon}, take the discarded tile, place it on your right along with any two arbitrary tiles from your hand. If anyone points out, just say you made a mistake. The only penalty you get for doing this is a dead hand, but you don't have a ready hand anyway. This can save you 500--1500 points. 

\bigskip
Second, an even more serious kind of dirty tactics is to call {\jap pon} when someone wins a hand by {\jap tsumo}. Again, you can do so without having the correct Pair to call {\jap pon}. Suppose

\end{boxnote}

\newpage
\begin{boxnote} \small
the Left player declares {\jap tsumo}. Then, you could call {\jap pon} on the tile just discarded by the Facing player, as long as it is within three seconds (and three seconds is a \emph{really} long time). 
The real beauty of this tactics is that, if the Left player has already revealed his hand, you could go on to claim that the Left player should get a {\jap chonbo} penalty for revealing the hand illegally! (I mean, if you have the nerves to do so...)
Even if he has not revealed his hand, he must have revealed the tile he drew to declare {\jap tsumo}. This means that you and the other two players will get to see which tile he is waiting for. 

\bigskip
You must have no shame to employ these tactics, but both of them are perfectly legal according to the current EMA rules. They can be easily prevented if the three-second rule is abolished. {\jap pon} should take precedence over {\jap chii} \emph{only} when calls are made simultaneously; otherwise, whoever calls first should take precedence, whether it is {\jap chii} or {\jap pon}. 
\end{boxnote}
\end{comment}

